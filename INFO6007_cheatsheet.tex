% INFO6007 cheatsheet (rebuilt, grouping bullets)
\documentclass[a4paper,landscape]{article}
\usepackage{xcolor}
\usepackage[top=.6mm,bottom=3.4mm,left=4.8mm,right=4mm]{geometry}
\usepackage{fontspec}
\usepackage{newunicodechar}
\usepackage{microtype}
\usepackage{multicol}
\usepackage{enumitem}
\setlist[itemize]{leftmargin=0pt,topsep=0pt, itemsep=0pt, parsep=0pt, label=\textbullet}
\setlength{\columnsep}{7pt}
\renewcommand{\familydefault}{\sfdefault}
\pagestyle{empty}
\defaultfontfeatures{Ligatures=TeX}
\setsansfont{Verdana}
\begin{document}
\fontsize{5pt}{5pt}\selectfont
\raggedright
\begin{multicols}{7}

\newcommand{\hlhighlight}[1]{%
    \begingroup
        \setlength{\fboxsep}{0pt}
        \colorbox{yellow}{\strut\textbf{#1}}%
    \endgroup
}
\definecolor{sectitlecolor}{HTML}{0050D0}
\definecolor{conceptcolor}{HTML}{008B8B} % teal color hex refined
\definecolor{procnamecolor}{HTML}{9B30FF}
\definecolor{categorycolor}{HTML}{228B22}
\newcommand{\sectitle}[1]{\textcolor{sectitlecolor}{\textbf{#1}}}
\newcommand{\concept}[1]{\textcolor{conceptcolor}{\textbf{#1}}}
\newcommand{\procname}[1]{\textcolor{procnamecolor}{\textbf{#1}}}
\newcommand{\category}[1]{\textcolor{categorycolor}{\textbf{#1}}}
\newcommand{\lbl}[1]{\textcolor{categorycolor}{\textbf{#1}}}
\newcommand{\lblnb}[1]{\textcolor{categorycolor}{#1}}

%%%%%%%%%%%%%%%%%%%%%%%%%%%%%%%%
%%%% INTRO / WEEK 1
%%%%%%%%%%%%%%%%%%%%%%%%%%%%%%%%

\hlhighlight{Introduction}

\sectitle{What is a Project?}
\begin{itemize}
\item \concept{PMBOK6e:} ``A temporary endeavor undertaken to create a unique product, service, or result.''
\end{itemize}

\sectitle{Key Characteristics:}
\begin{itemize}
\item \concept{Temporary:} Clear start and end point; not an ongoing operation.
\item \concept{Unique Outcome:} Produces something new (product, service, event, etc.).
\item \concept{Specific Objectives:} Aims to meet particular goals, requirements, or solve a problem.
\item \concept{Constraints:} Works within limitations of scope, time, cost, and resources.
\end{itemize}

\sectitle{Projects vs Operations}
\begin{itemize}
\item Projects = temporary and unique (change-focused).
\item Operations = ongoing and repetitive (run the business).
\end{itemize}

\sectitle{IT Projects}
\begin{itemize}
\item Technology-focused initiatives aimed at delivering a digital solution, system, or service.
\item Involve development, implementation, upgrade, or integration of IT systems.
\item Can be diverse in size, complexity, products, application area, and resource requirements.
\item Use diverse technologies that change rapidly; require specialized people.
\end{itemize}

\sectitle{Project Management}
\begin{itemize}
\item \concept{PMBOK6e:} Application of knowledge, skills, tools and techniques to project activities to meet requirements.
\end{itemize}

\sectitle{Key Elements (Constraints):}
\begin{itemize}
\item \concept{Scope:} What to achieve/deliverables.
\item \concept{Time:} How long/deadlines.
\item \concept{Cost:} Budget/resources.
\item \concept{Quality:} Meeting standards?
\item \concept{Risk:} What could go wrong?
\item \concept{Communication:} Information sharing.
\item \concept{Resources:} People, tools, technology.
\end{itemize}

\sectitle{Why IT Project Management is Critical?}
\begin{itemize}
\item Aligns technology with Business goals.
\item Delivers value on time and within budget.
\item Manages Complexity across teams.
\item Avoids scope creep and missed deadlines.
\item Reduces risks and improves accountability.
\item Improves Communication.
\item Handles change effectively.
\item \concept{Consequences of no PM:} Scope Creep, Miscommunication, Missed deadlines, Budget blowouts, Technical Failures, Stakeholder dissatisfaction.
\end{itemize}

\sectitle{Skills for Project Management}
\begin{itemize}
\item \category{Core PM skills:} Planning \& Scheduling, Scope management, Risk management, Budgeting \& cost control, Quality management, Documentation and Reporting.
\item \category{Soft skills:} Communication, Leadership, Conflict resolution, Negotiation, Decision making, Adaptability.
\item \category{Technical skills:} PM methodologies, Tools (Jira, Trello, Asana), Reporting.
\item \category{Strategic skills:} Stakeholder management, Business Acumen, Change management.
\item \category{Bonus traits:} Detail-oriented but Big picture thinkers, Calm under pressure, Proactive problem solvers, Empathetic Team Leaders.
\end{itemize}

\sectitle{Triple Constraints of Project Management}
\begin{itemize}
\item \concept{Scope:} Work required (features, functions, tasks).
\item \concept{Time:} Schedule or duration (deadlines, milestones).
\item \concept{Cost:} Budget (resources, labor, tools).
\item \concept{Interdependency:}
\item Scope increase –> Time and/or Cost increase.
\item Time reduced –> Cost increase or Scope decrease.
\item Budget reduced –> Scope reduce.
\end{itemize}

\sectitle{Project Success}
\begin{itemize}
\item Meeting or exceeding defined objectives, expectations, and stakeholder satisfaction across key dimensions.
\item Not just "on time, on budget", but delivering value.
\item \concept{Dimensions:} Time, Cost, Scope, Quality, Stakeholder Satisfaction, Business Value, Team Performance, Sustainability, User adoption, Alignment with strategic goals, Long term operational impact, Risk management and flexibility.
\item \concept{Misconceptions:} Finishing on time/budget is success (even if not usable/valuable); Meeting business goals is enough (ignoring team/stakeholders); All success is immediate (some impact is long-term).
\end{itemize}

\sectitle{Project vs Program vs Portfolio Management}
\sectitle{Project Management (Tactical Execution):}
\begin{itemize}
\item \concept{Focus:} Specific output (product/service).
\item \concept{Manager:} Project Manager.
\item \concept{Constraints:} Scope, Time, Budget.
\item \concept{Objective:} Deliver a successful project.
\end{itemize}

\sectitle{Program Management (Strategic Coordination):}
\begin{itemize}
\item Collection of related projects managed coordinately to achieve benefits not possible separately.
\item \concept{Manager:} Program Manager.
\item \concept{Focus:} Interdependencies, risks, strategic alignment.
\item \concept{Objective:} Deliver business value across multiple projects.
\end{itemize}

\sectitle{Portfolio Management (Enterprise Alignment):}
\begin{itemize}
\item Collection of programs/projects grouped to meet strategic business goals.
\item \concept{Manager:} Portfolio Manager or PMO.
\item \concept{Focus:} Investment decisions, resource allocation, strategic prioritization.
\item \concept{Objective:} Maximize value and alignment with organizational strategy.
\end{itemize}

\sectitle{Project Management Framework}
\begin{itemize}
\item Structured approach defining how projects are planned, executed, monitored, and completed.
\item \concept{Components:} Project Lifecycle, Processes and Knowledge Areas, Tools and Templates, Methodologies.
\item \concept{Why use:} Consistency, Reduces risk/increases predictability, Improves communication/accountability, Supports decision-making, Aligns with goals.
\item \category{PMBOK Framework:} 10 Knowledge Areas, 5 Process Groups (Initiating, Planning, Executing, Monitoring \& Controlling, Closing).
\end{itemize}

%%%%%%%%%%%%%%%%%%%%%%%%%%%%%%%%
%%%% WEEK 2 – METHODS
%%%%%%%%%%%%%%%%%%%%%%%%%%%%%%%%

\hlhighlight{PM Methodologies}

\sectitle{Project Lifecycle}
\begin{itemize}
\item Structured sequence of phases a project goes through.
\end{itemize}

\sectitle{Phases:}
\begin{itemize}
\item \concept{Initiation:} Define at high level, evaluate feasibility. Output: Approved project.
\item \concept{Planning:} Build comprehensive plan. Output: Approved PM Plan.
\item \concept{Execution:} Perform work. Output: Deliverables.
\item \concept{Monitoring \& Controlling:} Track performance, adjustments. Output: Status reports, corrections.
\item \concept{Closure:} Wrap up, deliver results, reflect. Output: Final report, lessons learned.
\end{itemize}

\sectitle{Project Management Methodologies}
\sectitle{Waterfall Model:}
\begin{itemize}
\item Linear, sequential. Each phase must complete before the next begins.
\item \concept{Phases:} Requirements –> System Design –> Implementation –> Testing –> Deployment –> Maintenance.
\item \category{Pros:} Simple, clear milestones, good for fixed scope/stable requirements, ideal for compliance-heavy projects.
\item \category{Cons:} Inflexible to changes, difficult to go back, delayed testing finds late problems.
\item \category{Best suited:} Stable requirements, regulatory/compliance projects, when documentation is critical.
\end{itemize}

\sectitle{Agile Methodology:}
\begin{itemize}
\item Flexible, iterative approach. Breaks project into small increments (sprints).
\item \category{4 Key Values:}
\item Individuals \& Interactions over processes and tools.
\item Working Software over comprehensive documentation.
\item Customer Collaboration over contract negotiation.
\item Responding to Change over following a plan.
\item \category{Best suited:} Evolving requirements, high uncertainty, need for rapid feedback.
\end{itemize}

\sectitle{Scrum:}
\begin{itemize}
\item Agile framework, time-boxed iterations (Sprints, 1-4 weeks).
\item \concept{Roles:} Product Owner (features/backlog), Scrum Master (coach), Scrum Team (do work).
\item \concept{Artifacts:} Product Backlog, Sprint Backlog, Increment.
\item \concept{Events:} Sprint Planning, Daily Stand up, Sprint Review, Sprint Retrospective.
\item \concept{Definition of Done (DoD):} Shared agreement on completion criteria.
\item \concept{Sprint Goal:} Single, coherent objective for the sprint.
\end{itemize}

\sectitle{Kanban:}
\begin{itemize}
\item Visual workflow management, continuous delivery.
\item \category{Key elements:}
\item Kanban Board — Visual board that tracks tasks through stages (e.g., To Do –> Doing –> Done).
\item Cards / User stories — Individual work items representing tasks or features.
\item WIP (Work In Progress) Limits — Constraints on how many items can be in a column to avoid overload.
\item Cycle Time — Time taken for a work item to move from start to finish.
\item \category{Features:}
\item Continuous flow of work rather than fixed-length iterations.
\item No specific required roles — teams can adopt Kanban with existing structure.
\item Work is pulled as capacity allows (pull-based system).
\item Emphasis on visual management to identify bottlenecks and improve flow.
\item \category{Best suited:} BAU/support work, continuous flow, service teams.
\end{itemize}

\sectitle{DevOps:}
\begin{itemize}
\item Integrates software development (Dev) and IT operations (Ops).
\item \concept{Goal:} Shorten development lifecycle, increase deployment frequency, deliver high-quality continuously.
\item \concept{Practices:} CI/CD, Automation, Infrastructure as Code (IaC), Monitoring.
\item \concept{Benefits:} Accelerated delivery, Automation, Collaboration, Reliability.
\end{itemize}

\hlhighlight{Requirement Gathering}
\begin{itemize}
\item Identifying, collecting, and documenting stakeholder needs/expectations.
\item \concept{Importance:} Meets stakeholder needs, Reduces scope creep/rework, Foundation for design/dev/test, Better planning/estimation.
\item \concept{Techniques:} Interviews, Surveys/Questionnaires, Workshops, Observation, Document Analysis.
\end{itemize}

\sectitle{Requirement Types}
\begin{itemize}
\item Functional requirements (what the system should do).
\item Non-functional requirements (performance, security, usability, reliability).
\end{itemize}

\sectitle{Common Requirement Issues}
\begin{itemize}
\item Ambiguous, conflicting, incomplete requirements –> risk of rework and scope creep.
\end{itemize}

\sectitle{Project Initiation / Defining}
\begin{itemize}
\item \concept{Purpose:} Clarity on what/why, Buy-in, Baseline.
\item \concept{Activities:} Identify need, Define goals, High-level scope, Identify stakeholders, Appoint PM, Develop Project Charter, Budget/Timeline, Risk/Feasibility, Alignment.
\item \category{Project Charter:} Formal document authorizing project and PM authority.
\item \concept{Purpose:} Vision, Aligns stakeholders, Roadmap, Reference.
\item \concept{Tools:} Expert Judgment, Business Case, Enterprise Environmental Factors, Organizational Process Assets.
\item \concept{Content:} Title, Dates, Background, Objectives, Success Criteria, Milestones, Budget, Roles (Sponsor, PM, Team).
\end{itemize}

\sectitle{Stakeholder Identification \& Engagement:}
\begin{itemize}
\item Identify anyone impacting/impacted by project.
\item Analyze (Power, Interest, Influence). Prioritize (Power/Interest Grid).
\item Engage \& Communicate strategies.
\item \concept{Challenges:} Unclear objectives, Lack of support, Poor engagement, Weak business case, Unrealistic constraints.
\item \concept{Tools (Stakeholder processes):} stakeholder analysis, power–interest grid, stakeholder engagement strategies.
\end{itemize}

%%%%%%%%%%%%%%%%%%%%%%%%%%%%%%%%
%%%% WEEK 3 – SCOPE & TIME
%%%%%%%%%%%%%%%%%%%%%%%%%%%%%%%%

\hlhighlight{Scope Management}
\begin{itemize}
\item \category{Scope Management Plan:} Defines how scope is defined, validated, and controlled.
\item \concept{Key Components:} Scope Statement, Management Approach, Requirement Collection, WBS, Validation Process, Control Process, Roles, Assumptions/Constraints.
\end{itemize}

\procname{6 Processes:}
\begin{itemize}
\item \lbl{Plan Scope Management:} Creating the plan.
\item \lbl{Collect Requirements:} Determining stakeholder needs. \lbl{Techniques:} Interviews, Workshops, Surveys, Prototyping, Observation. \lbl{Output:} Requirements traceability matrix.
\item \lbl{Define Scope:} Developing detailed description (deliverables, boundaries, acceptance criteria). \lbl{Output:} Project Scope Statement.
\item \lbl{Create WBS (Work Breakdown Structure):} Decomposing deliverables into smaller components (Work Packages).
\item \lbl{Validate Scope:} Formal acceptance of completed deliverables by customer/sponsor.
\item \lbl{Control Scope:} Monitoring status and managing changes to scope baseline.
\end{itemize}

\sectitle{Tools by Scope Process}
\begin{itemize}
\item \lblnb{Plan Scope Mgmt}: expert judgment, templates, organizational process assets.
\item \lblnb{Collect Requirements}: interviews, workshops, surveys, prototyping, observation, RTM.
\item \lblnb{Define Scope}: expert judgment, product analysis, alternatives analysis.
\item \lblnb{Create WBS}: decomposition, WBS templates, analogy, top-down/bottom-up, mind-mapping.
\item \lblnb{Validate Scope}: inspections, reviews, walkthroughs.
\item \lblnb{Control Scope}: variance analysis, change control system, trend analysis.
\end{itemize}

\sectitle{Scope Statement Details}
\begin{itemize}
\item Includes: Product scope description, Deliverables, Exclusions, Constraints, Assumptions, Success criteria.
\end{itemize}

\sectitle{Requirements Traceability Matrix (RTM)}
\begin{itemize}
\item Typical columns: ID, Description, Source, Priority, Acceptance Criteria, WBS Link, Test Case Link, Status.
\end{itemize}

\concept{Approaches:}
\begin{itemize}
\item Guidelines, Analogy, Top-Down, Bottom-Up, Mind-mapping.
\end{itemize}

\concept{Rules:}
\begin{itemize}
\item Unit of work in one place, sum of items below equals item above, one individual responsible, consistent with work performance.
\end{itemize}

\sectitle{Scope Management Issues}
\begin{itemize}
\item Scope creep, lack of stakeholder engagement, miscommunication, overambitious scope.
\item Mitigation: Clear scope statement, RTM, stakeholder sign-offs, formal change control.
\end{itemize}

\hlhighlight{Schedule (Time) Management}
\begin{itemize}
\item Planning, developing, managing, executing, and controlling project schedule.
\end{itemize}

\procname{6 Processes:}
\begin{itemize}
\item \lbl{Plan Schedule Management:} Define policy/procedures.
\item \lbl{Define Activities:} Break down WBS work packages into specific tasks. \lbl{Output:} Activity list, attributes, Milestone list.
\item \lbl{Sequence Activities:} Determine logical order and dependencies.
\item \lbl{Estimate Activity Durations:} Predict time required. \concept{Techniques:} Expert judgment, Analogous (historical data), Parametric (statistical), Bottom-up (detailed aggregation), Three-point (PERT).
\item \lbl{Develop Schedule:} Integrate durations, sequences, resources.
\item \lbl{Control Schedule:} Monitor progress, manage changes. Uses EVM, Variance Analysis (SV, SPI).
\end{itemize}

\sectitle{Tools by Schedule Process}
\begin{itemize}
\item \lblnb{Plan Schedule Mgmt}: expert judgment, templates, PMIS.
\item \lblnb{Define Activities}: decomposition, rolling wave planning, templates.
\item \lblnb{Sequence Activities}: network diagrams (AON/AOA/PDM), dependency types, leads \& lags.
\item \lblnb{Estimate Durations}: expert judgment, analogous, parametric, bottom-up, 3-point/PERT.
\item \lblnb{Develop Schedule}: CPM, Gantt chart, schedule compression (fast tracking, crashing), resource leveling, PMIS.
\item \lblnb{Control Schedule}: EVM (SV, SPI), variance analysis, schedule forecasts, performance reviews.
\end{itemize}

\sectitle{Schedule Baseline}
\begin{itemize}
\item Approved version of the schedule used to track performance and calculate SV/SPI.
\end{itemize}

\sectitle{Network Diagrams}
\begin{itemize}
\item \lbl{AON (Activity on Node):} activities represented by nodes (boxes) and arrows show dependencies.
\item \lbl{AOA (Activity on Arrow):} activities as arrows and nodes represent events; uses dummy activities.
\item \lbl{PDM (Precedence Diagramming Method):} defines different dependency types.
\item \concept{Dependency Types:} Finish-to-Start (FS), Start-to-Start (SS), Finish-to-Finish (FF), Start-to-Finish (SF).
\item \lbl{Lead Time:} Accelerate schedule (overlap).
\item \lbl{Lag Time:} Delay/waiting period.
\end{itemize}

\sectitle{Dependency Examples}
\begin{itemize}
\item FS: Coding must finish before Testing starts.
\item SS: UI design and UX research can start in parallel.
\item FF: Deployment and rollback preparations finish together.
\item SF: Rare, e.g. maintenance finishes when new system starts.
\end{itemize}

\sectitle{Lead/Lag Examples}
\begin{itemize}
\item Lead: Testing starts 2 days before Coding finishes.
\item Lag: Wait 3 days after concrete pour before inspection.
\end{itemize}

\sectitle{Estimating Activity Durations}
\begin{itemize}
\item \concept{Techniques:} Expert judgment, Analogous, Parametric, Bottom-up, Three-point (PERT).
\item \lbl{PERT} Formula: (Optimistic + 4 * Most Likely + Pessimistic) / 6.
\end{itemize}

\sectitle{Schedule Tools / Techniques}
\begin{itemize}
\item \lbl{CPM (Critical Path Method):} Identifies longest sequence of dependent activities (shortest project duration). Zero float/slack.
\item \lbl{Gantt Chart:} Visual bar chart of schedule.
\item \lbl{Schedule Compression:} Fast Tracking (parallel - risk), Crashing (add resources - cost).
\end{itemize}

%%%%%%%%%%%%%%%%%%%%%%%%%%%%%%%%
%%%% WEEK 4 – COST
%%%%%%%%%%%%%%%%%%%%%%%%%%%%%%%%

\hlhighlight{Cost Management Plan}

\sectitle{Project Cost Management}
\begin{itemize}
\item Process of estimating, budgeting, and controlling costs.
\end{itemize}

\concept{Types of Costs:}
\begin{itemize}
\item \lbl{Tangible vs Intangible:} Easy vs difficult to measure in \$.
\item \lbl{Direct vs Indirect:} Directly related to project vs overhead/admin.
\item \lbl{Sunk:} Money already spent (ignore for future decisions).
\end{itemize}

\concept{Reserves:}
\begin{itemize}
\item \lbl{Contingency:} For known unknowns (in cost baseline).
\item \lbl{Management:} For unknown unknowns (added to baseline for total budget).
\end{itemize}

\procname{4 Processes}
\begin{itemize}
\item \lbl{Plan Cost Management:} Define policies/procedures.
\item \lbl{Estimate Costs:} Approximation of resources needed.
\item \lbl{Determine Budget:} Aggregating estimates to authorized cost baseline.
\item \lbl{Control Costs:} Monitor status, update costs, manage baseline changes.
\end{itemize}

\sectitle{Tools by Cost Process}
\begin{itemize}
\item \lblnb{Plan Cost Mgmt}: expert judgment, analytical techniques, cost management plan templates.
\item \lblnb{Estimate Costs}: analogous, parametric, bottom-up, 3-point, expert judgment, reserve analysis.
\item \lblnb{Determine Budget}: cost aggregation (sum WBS), reserve analysis, historical information, funding limit reconciliation.
\item \lblnb{Control Costs}: EVM (CV, SV, CPI, SPI, EAC, ETC), variance analysis, performance reviews, forecasting.
\end{itemize}

\sectitle{Estimate Accuracy}
\begin{itemize}
\item Rough Order of Magnitude (ROM): approx. -25\% to +75\%.
\item Definitive estimate: approx. -10\% to +10\%.
\end{itemize}

\sectitle{Cost Estimation Techniques}
\begin{itemize}
\item Analogous, Parametric, Bottom-up, Three-point, Expert judgment.
\end{itemize}

\sectitle{IT Cost Estimation Notes}
\begin{itemize}
\item Influenced by requirement clarity, tech uncertainty, integration complexity, team skill.
\item Common errors: ignoring training, maintenance, security/compliance, integration, vendor lock-in.
\item Bottom-up example: 40h * \$100/h = \$4000.
\end{itemize}

\sectitle{Earned Value Management (EVM)}
\begin{itemize}
\item Integrates scope, schedule, and cost to measure performance.
\item \concept{Core Components:}
\item \lbl{Planned Value (PV):} Budgeted cost of work scheduled.
\item \lbl{Earned Value (EV):} Budgeted cost of work actually completed.
\item \lbl{Actual Cost (AC):} Real cost incurred.
\end{itemize}

\concept{Variance Analysis:}
\begin{itemize}
\item \lbl{Cost Variance (CV):} EV - AC. \\(+ Under budget, - Over budget).
\item \lbl{Schedule Variance (SV):} EV - PV. \\(+ Ahead, - Behind).
\end{itemize}

\concept{Performance Indexes:}
\begin{itemize}
\item \lbl{Cost Performance Index (CPI):} EV / AC. \\(>1 Efficient, <1 Overrun).
\item \lbl{Schedule Performance Index (SPI):} EV / PV. \\(>1 Ahead, <1 Behind).
\end{itemize}

\concept{Forecasting:}
\begin{itemize}
	\item \lbl{Estimate at Completion (EAC):} Expected total cost based on current trends. BAC / CPI.
	\item \lbl{Estimate to Complete (ETC):} Remaining cost to finish. Common forms: ETC = EAC - AC.
	\item \lbl{Budget at Completion (BAC):} Total planned budget from cost baseline.
\end{itemize}

\sectitle{Traditional vs Agile Cost Management}
\begin{itemize}
\item \lbl{Traditional:} Fixed budget/baseline upfront, Detailed estimation (WBS), EVM works well.
\item \lbl{Agile:} Budget per sprint/release, Rolling-wave estimates, EVM difficult (scope changes), Focus on value (story points).
\item \lbl{Practices:} Fund increments, Burn charts, Agile/Lightweight EVM, Contingency reserves.
\end{itemize}

%%%%%%%%%%%%%%%%%%%%%%%%%%%%%%%%
%%%% WEEK 5 – QUALITY
%%%%%%%%%%%%%%%%%%%%%%%%%%%%%%%%

\hlhighlight{Quality Management}

\sectitle{Quality}
\begin{itemize}
\item Degree to which inherent characteristics fulfill requirements.
\item \lbl{Dimensions in IT:} Functionality, Reliability, Performance, Usability, Maintenance, Security.
\item \lbl{Perspectives:} Conformance to Requirements, Fitness for Use, Customer Satisfaction.
\end{itemize}

\sectitle{Project Quality Management}
\begin{itemize}
\item Managing quality of deliverables (product) and processes.
\item \lbl{Principles:} Customer focus, Prevention over Inspection, Continuous Improvement, Fact-based decision making.
\end{itemize}

\procname{3 Processes}
\begin{itemize}
\item \lbl{Plan Quality Management:} Identify standards/requirements, how to comply.
\item \lbl{Manage Quality (QA):} Translating plan into executable activities (Process-focused).
\item \lbl{Control Quality (QC):} Monitoring deliverables meet standards (Product-focused).
\end{itemize}

\sectitle{Tools by Quality Process}
\begin{itemize}
\item \lblnb{Plan QM}: cost–benefit analysis, cost of quality, benchmarking, flowcharts, quality metrics, standards (ISO).
\item \lblnb{Manage Quality (QA)}: quality audits, process analysis, design of experiments, checklists, PDCA.
\item \lblnb{Control Quality (QC)}: inspection, 7QC tools (control chart, Pareto, fishbone, histogram, checksheet, scatter, flowchart), sampling.
\end{itemize}

\sectitle{Verification vs Validation}
\begin{itemize}
\item Verification: Are we building the product right? (meets specs).
\item Validation: Are we building the right product? (meets user needs).
\end{itemize}

\sectitle{Cost of Quality (COQ)}
\begin{itemize}
\item Prevention costs, Appraisal costs, Internal failure costs, External failure costs.
\end{itemize}

\sectitle{Quality Standards \& Frameworks}
\begin{itemize}
\item ISO 9001 (Quality management systems).
\item ISO/IEC 27001 (Information security).
\item CMMI (Process maturity).
\end{itemize}

\sectitle{Agile Quality Perspective}
\begin{itemize}
\item Built-in quality, continuous testing, whole-team responsibility.
\end{itemize}

\sectitle{Tools and Techniques}
\begin{itemize}
\item \lbl{Cost-Benefit Analysis:} Cost vs Benefits (ROI).
\item \lbl{Cause and Effect (Fishbone) Diagram:} Root cause analysis.
\item \lbl{Control Chart:} Process performance over time vs limits.
\item \lbl{Checksheet:} Data collection tally.
\item \lbl{Flowcharts:} Process mapping.
\item \lbl{Scatter Diagram:} Correlation.
\item \lbl{Histogram:} Frequency distribution.
\item \lbl{Pareto Chart:} 80/20 rule (80\% problems from 20\% causes).
\end{itemize}

\sectitle{PDCA Cycle}
\begin{itemize}
\item Plan –> Do –> Check –> Act (continuous improvement loop).
\end{itemize}

%%%%%%%%%%%%%%%%%%%%%%%%%%%%%%%%
%%%% WEEK 6 – RESOURCES
%%%%%%%%%%%%%%%%%%%%%%%%%%%%%%%%

\hlhighlight{Resource Management}

\sectitle{Project Resource Management}
\begin{itemize}
\item Identifying, acquiring, and managing resources (human and physical).
\item \lbl{Importance:} Specialized roles, Shared resources, Planning lead times.
\end{itemize}

\procname{6 Processes}
\begin{itemize}
\item \lbl{Plan Resource Management:} Define estimating, acquiring, managing resources.
\item \lbl{Estimate Activity Resources:} Determine types/quantities needed.
\item \lbl{Acquire Resources:} Obtaining team/physical resources.
\item \lbl{Develop Team:} Improving competencies, interaction, environment.
\item \lbl{Manage Team:} Tracking performance, feedback, resolving issues.
\item \lbl{Control Resources:} Monitoring physical resource usage vs plan.
\end{itemize}

\sectitle{Tools by Resource Process}
\begin{itemize}
\item \lblnb{Plan Resource Mgmt}: RACI, RBS, org charts, resource management plan.
\item \lblnb{Estimate Activity Resources}: expert judgment, bottom-up estimating, RBS, alternatives analysis.
\item \lblnb{Acquire Resources}: negotiation, pre-assignment, acquisition from functional managers.
\item \lblnb{Develop Team}: training, team-building, Tuckman model, motivation theories (Maslow, Herzberg), recognition \& rewards.
\item \lblnb{Manage Team}: leadership styles, conflict management styles, performance appraisals, feedback.
\item \lblnb{Control Resources}: resource histogram, resource leveling, inspections, variance analysis.
\end{itemize}

\sectitle{Resource Structures \& Analysis}
\begin{itemize}
\item \concept{RBS:} Resource Breakdown Structure (Human, Equipment, Materials, etc.).
\item Resource histogram: shows workload/utilisation over time.
\item Resource leveling: adjust task timing to remove over-allocation (may extend schedule).
\end{itemize}

\hlhighlight{Human Resources}
\begin{itemize}
\item \lbl{Motivation:} Intrinsic vs Extrinsic.
\item \concept{Tuckman's Model:}(Group development) Forming, Storming, Norming, Performing, Adjourning.
\item \concept{RACI:} Responsible (does work), Accountable (owns result), Consulted (provides input), Informed (kept updated).
\end{itemize}

\sectitle{Motivation Theories}
\begin{itemize}
\item \lbl{Maslow:} physiological –> safety –> belonging –> esteem –> self-actualisation.
\item \lbl{Herzberg:} hygiene factors (salary, policy, conditions) vs motivators (recognition, achievement, growth).
\end{itemize}

\sectitle{Conflict Management Styles}
\begin{itemize}
\item Avoid, Accommodate, Compete, Compromise, Collaborate.
\end{itemize}

%%%%%%%%%%%%%%%%%%%%%%%%%%%%%%%%
%%%% WEEK 7 – RISK
%%%%%%%%%%%%%%%%%%%%%%%%%%%%%%%%

\hlhighlight{Risk Management}

\sectitle{Project Risk Management}
\begin{itemize}
\item Managing uncertain events (positive/negative) affecting objectives.
\item \lbl{Types:} Threats (Negative), Opportunities (Positive).
\item \lbl{Categories:} Market, Financial, Technology, People, Structure/Process.
\end{itemize}

\procname{7 Processes (Based on slides structure)}
\begin{itemize}
\item \lbl{Plan Risk Management:} Define approach.
\item \lbl{Identify Risks:} Determining risks/characteristics. \lbl{Output:} Risk Register.
\item \lbl{Perform Qualitative Risk Analysis:} Prioritizing by Probability x Impact.
\item \lbl{Perform Quantitative Risk Analysis:} Numerical analysis of effect.
\item \lbl{Plan Risk Responses:} Strategies.
\item \lbl{Implement Risk Responses / Monitor:} Track and apply responses.
\item \lbl{Monitor and Control Risks:} Tracking, monitoring, identifying new, evaluating effectiveness.
\end{itemize}

\sectitle{Tools by Risk Process}
\begin{itemize}
\item \lblnb{Plan Risk Mgmt}: risk management plan templates, expert judgment, stakeholder analysis.
\item \lblnb{Identify Risks}: brainstorming, interviews, checklists, SWOT, risk register, RBS.
\item \lblnb{Qualitative RA}: probability–impact matrix, risk categorisation, risk urgency, risk ranking.
\item \lblnb{Quantitative RA}: EMV, decision trees, Monte Carlo simulation, sensitivity analysis.
\item \lblnb{Plan Risk Responses}: strategy selection (avoid/mitigate/transfer/accept; exploit/enhance/share/accept), contingency planning.
\item \lblnb{Implement/Monitor Risks}: risk register updates, risk audits, risk reviews, risk burndown chart (Agile), issue logs.
\end{itemize}

\sectitle{Risk Response Strategies}
\begin{itemize}
\item \concept{Threats:} Avoid, Mitigate, Transfer, Accept, Escalate.
\item \concept{Opportunities:} Exploit, Enhance, Share, Accept, Escalate.
\end{itemize}

\sectitle{Risk Appetite \& Utility}
\begin{itemize}
\item Risk-averse, risk-neutral, risk-seeking.
\item \lbl{Utility curves}: Averse (concave), Neutral (linear), Seeking (convex).
\end{itemize}

\sectitle{Risk Structures}
\begin{itemize}
\item \concept{Risk Register:} Document recording risks, causes, owners, responses; living document.
\item \concept{RBS:} Risk Breakdown Structure (Technical, People, Process, Organizational, External).
\end{itemize}

\sectitle{Qualitative vs Quantitative Analysis}
\begin{itemize}
\item Qualitative: probability–impact matrix, risk ranking.
\item Quantitative: EMV, Monte Carlo simulation, decision trees.
\end{itemize}

\sectitle{Monte Carlo \& EMV}
\begin{itemize}
\item Monte Carlo: assign distributions to uncertain variables; simulate many iterations; get probability distribution of outcomes.
\item EMV (Expected Monetary Value): sum (Probability * Outcome).
\end{itemize}

\sectitle{Agile Risk Management}
\begin{itemize}
\item Iterative reviews, Risk-based backlog prioritization, Risk Burn-Down Charts, Collaborative ownership.
\end{itemize}

%%%%%%%%%%%%%%%%%%%%%%%%%%%%%%%%
%%%% WEEK 8 – PROCUREMENT
%%%%%%%%%%%%%%%%%%%%%%%%%%%%%%%%

\hlhighlight{Procurement Management}

\sectitle{Project Procurement Management}
\begin{itemize}
\item Acquiring goods, services, results from outside project team.
\end{itemize}

\procname{3 Processes:}
\begin{itemize}
\item \lbl{Plan Procurement Management:} Documenting decisions, approach, sellers.
\item \lbl{Conduct Procurements:} Obtaining responses, selecting sellers, awarding contracts.
\item \lbl{Control Procurements:} Managing relationships, monitoring performance, changes.
\end{itemize}

\sectitle{Tools by Procurement Process}
\begin{itemize}
\item \lblnb{Plan Procurement}: make-or-buy analysis, market research, SOW development, contract type selection, expert judgment.
\item \lblnb{Conduct Procurements}: RFP/RFQ/RFI, bidder conferences, weighted scoring models, advertising, proposal evaluation techniques.
\item \lblnb{Control Procurements}: SLA monitoring, performance reviews, contract change control system, inspections \& audits, claims administration, payment systems.
\end{itemize}

\sectitle{Documents \& Concepts}
\begin{itemize}
\item \lbl{Make-or-Buy Analysis:} Decide internal vs external.
\item \lbl{Statement of Work (SOW):} Description of work ("what and how").
\item \lbl{RFP:} (Proposal - solution focused), \lbl{RFQ:} (Quotation - price focused).
\item \lbl{RFI:} Request for Information (high-level capabilities).
\item Contract, SLA (Service Level Agreement) as key legal/ performance documents.
\end{itemize}

\sectitle{Contract Types}
\begin{itemize}
\item Fixed-price: vendor bears more cost risk.
\item Time \& Materials (T\&M): buyer bears cost risk; flexible scope.
\item Cost-reimbursable: buyer reimburses costs plus fee; shared risk.
\end{itemize}

\sectitle{Vendor Selection \& Evaluation}
\begin{itemize}
\item Weighted scoring model: criteria + weights + vendor scores –> total score.
\item Criteria: cost, technical capability, security, SLA terms, experience, cultural fit.
\end{itemize}

\sectitle{Contract Administration}
\begin{itemize}
\item Monitor performance vs SLA, manage contract changes, handle disputes, accept deliverables.
\end{itemize}

\hlhighlight{Outsourcing}
\begin{itemize}
\item Hiring external vendor to perform project tasks/deliverables instead of doing them in-house.
\item Used for cost reduction, access to expertise, faster delivery.
\end{itemize}

\sectitle{Why Outsource?}
\begin{itemize}
\item Cost efficiency (labour/infrastructure savings).
\item Access specialised skills (cloud, AI, security).
\item Focus on core business.
\item Risk sharing with vendor.
\end{itemize}

\sectitle{Types of Outsourcing}
\begin{itemize}
\item Complete project outsourcing (LC full handle).
\item Partial outsourcing (specific components/tasks).
\item BPO (non-core processes: helpdesk, payroll).
\item IT/Technical outsourcing (cloud, SW dev, cybersecurity).
\end{itemize}

\sectitle{Outsourcing Models}
\begin{itemize}
\item Onshore (same country), Nearshore (near timezone), Offshore (low-cost distant).
\item Hybrid (mix internal/external), Multisource (multiple vendors).
\item Insourcing = work done internally.
\end{itemize}

\sectitle{Benefits}
\begin{itemize}
\item Lower cost, global talent, faster delivery.
\item Reduce internal workload; flexibility; access new tech.
\end{itemize}

\sectitle{Disadvantages}
\begin{itemize}
\item Communication/time-zone issues; quality risks.
\item Security/privacy concerns; vendor dependency.
\item Hidden coordination/contract costs.
\end{itemize}

\sectitle{Governance of Outsourcing}
\begin{itemize}
\item Ensures outsourced work aligns with objectives, meets quality, and manages risk.
\item Key: strategic alignment, contract management, clear roles/responsibilities.
\end{itemize}

\sectitle{Integration with Procurement}
\begin{itemize}
\item Aligns outsourced work with project goals; reduces schedule/cost/quality risks.
\item Link contracts to milestones, use shared PM tools, communicate continuously, manage risks proactively.
\end{itemize}

%%%%%%%%%%%%%%%%%%%%%%%%%%%%%%%%
%%%% WEEK 9 – LEADERSHIP & GOV
%%%%%%%%%%%%%%%%%%%%%%%%%%%%%%%%
\hlhighlight{Leadership}
\begin{itemize}
\item Inspiring/guiding team,stakeholders to objectives
\item \lbl{vs Management:} People/Vision/Influence vs Process/Tasks/Control.
\item \lbl{Styles:} Transformational, Transactional, Autocratic, Laissez-Faire, Adaptive/Agile, Participative, Democratic
\end{itemize}

\sectitle{Leader vs Manager}
\begin{itemize}
\item Leader: vision, change, people focus.
\item Manager: execution, process, control focus.
\end{itemize}

\sectitle{Adaptive / Situational Leadership}
\begin{itemize}
\item Adjusts style based on team maturity, complexity, and context.
\end{itemize}

\hlhighlight{Communication Management}
\begin{itemize}
\item 3 Processes:
\item \lbl{Plan:} Needs, methods, roles.
\item \lbl{Manage:} Distribute, retrieve info.
\item \lbl{Monitor:} Ensure effectiveness.
\end{itemize}

\sectitle{Tools by Communication Process}
\begin{itemize}
\item \lblnb{Plan Communications}: communication requirements analysis, stakeholder analysis, choice of channels (email, meetings, dashboards), comms plan templates.
\item \lblnb{Manage Communications}: RAG reports, dashboards, RAID logs, meetings, presentations, collaboration tools.
\item \lblnb{Monitor Communications}: feedback, observations, stakeholder engagement assessments, surveys.
\end{itemize}

\sectitle{Communication Plan Contents}
\begin{itemize}
\item Stakeholder, information needs, format, frequency, owner.
\item Tools: RAG reports, dashboards, RAID logs.
\end{itemize}

\hlhighlight{Governance}
\begin{itemize}
\item Framework of policies, roles, decision-making structures.
\item \lbl{Structure:} Steering Committees (Strategic guidance), PMO (Standardization/Support), Project Sponsors (Champion/Accountability).
\end{itemize}

\sectitle{Governance Mechanisms}
\begin{itemize}
\item Stage/phase-gate reviews, escalation paths, KPIs, audits, formal reporting structures.
\end{itemize}

\sectitle{Steering Committee Responsibilities}
\begin{itemize}
\item Approve business case/charter, set strategic objectives, approve major changes, resolve escalated issues, ensure benefits realisation.
\end{itemize}

\sectitle{PMOffice Functions}
\begin{itemize}
\item Standardise methods/templates, provide training, support project managers, portfolio reporting, tool support.
\end{itemize}

\hlhighlight{Alignment}
\begin{itemize}
\item \concept{Definition:} Goals, activities, and outcomes of a project are in harmony with the business, stakeholder and organisation.
\item \concept{Why it matters:} Maximizes Business Value, Reduces Wasted Effort, Improves Stakeholder Satisfaction, Enhances Decision-Making, Boosts Adoption
\item \concept{How to achieve:} Map deliverables to strategic objectives, engage stakeholders early, embed alignment checks in governance (stage gates/steering committee), use KPIs/OKRs, and review alignment regularly.
\end{itemize}
\sectitle{Alignment Frameworks}
\begin{itemize}
\item Strategic Alignment Model (SAM) – align business and IT strategy.
\item Balanced Scorecard – financial, customer, internal process, learning/growth perspectives.
\item COBIT – IT governance framework.
\item TOGAF – enterprise architecture framework (align IT architecture with business).
\end{itemize}

%%%%%%%%%%%%%%%%%%%%%%%%%%%%%%%%
%%%% WEEK 10 – INTEGRATION
%%%%%%%%%%%%%%%%%%%%%%%%%%%%%%%%

\hlhighlight{Integration Management}

\sectitle{Project Integration Management}
\begin{itemize}
\item Coordination of all processes/activities. "Glue".
\item Unique knowledge area that balances and integrates all others (scope, time, cost, quality, etc.).
\end{itemize}

\procname{7 Processes:}
\begin{itemize}
\item \lbl{Develop Project Charter:} Authorization.
\item \lbl{Develop Project Management Plan:} Consolidate subsidiary plans.
\item \lbl{Direct and Manage Project Work:} Execute.
\item \lbl{Manage Project Knowledge:} Lessons learned.
\item \lbl{Monitor and Control Project Work:} Track performance.
\item \lbl{Perform Integrated Change Control:} Review/approve changes (CCB).
\item \lbl{Close Project or Phase:} Finalize.
\end{itemize}

\sectitle{Tools by Integration Process}
\begin{itemize}
\item \lblnb{Develop Charter}: expert judgment, facilitation workshops, business case, EEFs/OPAs.
\item \lblnb{Develop PM Plan}: expert judgment, meetings, PMIS, facilitation techniques.
\item \lblnb{Direct \& Manage Work}: PMIS, issue log, meetings, change requests.
\item \lblnb{Manage Project Knowledge}: lessons learned register, knowledge repositories, communities of practice.
\item \lblnb{Monitor \& Control Work}: data analysis (trend/variance), PMIS, performance reports.
\item \lblnb{Integrated Change Control}: change control system, CCB meetings, impact analysis.
\item \lblnb{Close Project/Phase}: document analysis, lessons learned meeting, final product/phase acceptance.
\end{itemize}

\sectitle{Key Tools \& Artefacts}
\begin{itemize}
\item \lbl{Project Management Information Systems (PMIS):} e.g., MS Project, Jira, Asana, Trello — supports scheduling, resource management, issue tracking, and reporting.
\item \lbl{Dashboards:} Real-time views of progress, risks, costs, and resource utilisation (visual RAG indicators, burn-downs, KPI widgets).
\item \lbl{Change Control Boards (CCB):} Formal groups that review, approve/reject, and prioritise change requests; coordinate impact analysis and communication.
\item \lbl{Documentation \& Knowledge Repositories:} Wikis, Confluence, SharePoint for storing lessons learned, templates, SOPs, decision logs, and best practices.
\end{itemize}

\sectitle{Project Management Plan Contents}
\begin{itemize}
\item Integrates all subsidiary plans (scope, schedule, cost, quality, resource, comms, risk, procurement, stakeholder) plus baselines and configuration management plan.
\end{itemize}

\sectitle{Integrated Change Control Flow}
\begin{itemize}
\item Change request –> impact analysis (scope/time/cost/risk) –> CCB decision –> update baselines –> communicate.
\end{itemize}

\sectitle{Knowledge Management}
\begin{itemize}
\item Explicit knowledge (documents, databases) vs Tacit knowledge (experience, skills).
\item Lessons learned register updated throughout; becomes organisational process asset (OPA).
\end{itemize}

\hlhighlight{Uniqueness of IT Projects}
\begin{itemize}
\item \concept{IT projects:} Complex, rapidly changing, stakeholder-heavy, and high-risk — require iterative delivery, continuous integration/testing, and active risk/requirements management to avoid costly rework.
\end{itemize}

%%%%%%%%%%%%%%%%%%%%%%%%%%%%%%%%
%%%% WEEK 11/12 – ITSM
%%%%%%%%%%%%%%%%%%%%%%%%%%%%%%%%

\hlhighlight{ITSM and Processes}

\sectitle{IT Service Management (ITSM)}
\begin{itemize}
\item Policies/processes to design, deliver, manage, and improve IT services.
\item \lbl{Goals:} Business alignment, Quality, Efficiency, Customer Satisfaction, Stability.
\end{itemize}

\sectitle{Products vs Services}
\begin{itemize}
\item \lbl{Product:} Tangible output (e.g. app, device) with clear ownership.
\item \lbl{Service:} Ongoing delivery that enables outcomes without customer owning all risks/costs.
\end{itemize}

\sectitle{Service Relationship Model}
\begin{itemize}
\item \lbl{Service Provider:} Organisation delivering the service.
\item \lbl{Service Consumer:} Users/customers receiving value.
\item \lbl{Co-creation of value:} Value realised only via interaction of provider + consumer (use, adoption, correct configuration).
\end{itemize}

\sectitle{Role of Information in Competitive Advantage}

Information is a strategic resource; enables efficiency, innovation, decision quality, customer loyalty, and trust.
\begin{itemize}
\item \lbl{Operational}: Streamlines processes; reduces cost — Efficiency
\item \lbl{Strategic}: Guides competitive positioning — Advantage
\item \lbl{Innovative}: Enables new products/services — Differentiation
\item \lbl{Managerial}: Supports data-driven decisions — Accuracy
\item \lbl{Customer}: Enhances UX; personalised services — Loyalty
\item \lbl{Governance}: Ensures ethical, compliant, secure use — Trust
\end{itemize}

\sectitle{ITIL v4.0}
\begin{itemize}
\item Framework for modern ITSM; flexible and value-driven.
\item \lbl{Service:} Means of enabling value co-creation by facilitating outcomes without customer managing specific costs/risks.
\item \lbl{Value:} Perceived benefits, utility and worth (subjective, contextual).
\item \lbl{Service Value System (SVS):} All components working together:
\begin{itemize}
  \item Guiding Principles
  \item Governance
  \item Service Value Chain (SVC)
  \item Practices (34)
  \item Continual Improvement
\end{itemize}
\end{itemize}

\sectitle{Guiding Principles (7)}
\begin{itemize}
\item Focus on value
\item Start where you are
\item Progress iteratively with feedback
\item Collaborate and promote visibility
\item Think and work holistically
\item Keep it simple and practical
\item Optimize and automate
\end{itemize}

\sectitle{Guiding Principle Examples}
\begin{itemize}
\item \lbl{Focus on value:} Measure outcomes from user/business perspective (uptime, satisfaction, adoption).
\item \lbl{Start where you are:} Reuse existing tools/data/processes where effective instead of rebuilding everything.
\item \lbl{Progress iteratively:} Deliver improvements in small increments with rapid feedback.
\item \lbl{Think holistically:} Consider end-to-end value streams (people, process, tech, partners).
\end{itemize}

\sectitle{Practices}
\begin{itemize}
\item \concept{Practice:} Set of organisational resources (people, processes, information, tools, partners) designed to perform work/achieve an objective.
\item \lbl{Categories (34 total):}
\begin{itemize}
  \item \category{General Mgmt (14):} Strategy, Info Security, Risk, Knowledge, Project Mgmt, Relationship Mgmt, etc.
  \item \category{Service Mgmt (17):} Incident, Problem, Change Enablement, Service Level Mgmt, Service Desk, Service Request, Asset Mgmt, Continuity, etc.
  \item \category{Technical Mgmt (3):} Deployment, Infrastructure \& Platform, Software Development \& Mgmt.
\end{itemize}
\end{itemize}

\sectitle{Key Service Practices (Ops Focus)}
\begin{itemize}
\item \lbl{Incident Management:} Restore normal service ASAP, minimising business impact (does \emph{not} necessarily fix root cause).
\item \lbl{Problem Management:} Identify/analyse root causes of incidents to prevent recurrence; maintain \emph{Known Error} records and workarounds.
\item \lbl{Change Enablement:} Ensure changes are assessed, approved, scheduled and implemented with minimal risk.
\begin{itemize}
  \item \lbl{Change types:} Standard (pre-approved, low risk, repeatable), Normal (assessed, may need CAB), Emergency (fast-track for urgent high-priority).
\end{itemize}
\item \lbl{Service Level Management (SLM):} Define, negotiate, monitor and report on SLAs (availability, response time, resolution time).
\item \lbl{Service Desk:} SPOC between users and IT; logs incidents/requests, resolves or escalates, keeps users informed.
\item \lbl{Continual Improvement:} Identify and implement improvements across services, practices and value streams.
\end{itemize}

\sectitle{Additional Service Practices Detail}
\begin{itemize}
\item \lbl{Service Request Management:}
\begin{itemize}
  \item Handle low-risk, standard requests (access, info, password reset).
  \item \concept{Diff vs Incident:} Incident = unplanned interruption/issue; Service Request = planned, pre-approved ask.
\end{itemize}
\item \lbl{Knowledge Management:}
\begin{itemize}
  \item Create/maintain knowledge base articles, FAQs, troubleshooting guides.
  \item Support Known Error Database (KEDB) for faster incident/problem resolution.
\end{itemize}
\item \lbl{Service Continuity Management:} Ensure services can be recovered after major disruptions (links with DR/BCP).
\end{itemize}

\sectitle{Continual Improvement – Simple Cycle}
\begin{itemize}
\item Identify improvement opportunity (data from incidents, problems, SLAs, feedback).
\item Assess and prioritise (impact vs effort/value).
\item Plan change (define target, owner, actions, measures).
\item Implement improvement.
\item Measure and review results; standardise successful changes.
\end{itemize}

\sectitle{Service Value Chain (SVC)}
\begin{itemize}
\item Operating model converting demand to value via 6 interconnected activities:
\item Plan, Improve, Engage, Design \& Transition, Obtain/Build, Deliver \& Support.
\item Each \emph{value stream} uses a tailored combination of these activities.
\end{itemize}

\sectitle{Value Streams}
\begin{itemize}
\item Specific paths through SVC activities for a given outcome.
\item \lbl{Example – Incident Resolution:} Engage (log) –> Deliver/Support (diagnose) –> Obtain/Build (fix) –> Design \& Transition (deploy) –> Deliver/Support (verify/close) –> Improve (analyse trends).
\end{itemize}

\sectitle{ITIL-aligned Tools (e.g. ServiceNow)}
\begin{itemize}
\item Cloud ITSM platforms automate and integrate practices:
\item Incident, Problem, Change, Service Request workflows.
\item SLA tracking, dashboards, reporting.
\item CMDB / asset/configuration records.
\item Knowledge base and self-service portal.
\end{itemize}

\sectitle{Automation \& AI in ITSM}
\begin{itemize}
\item \lbl{Goals:} Efficiency, speed, consistency, predictive capability.
\item \lbl{Applications:}
\begin{itemize}
  \item Chatbots/self-service for common incidents/requests.
  \item Auto-routing and prioritisation based on impact/urgency.
  \item Predictive analytics for recurring issues (Problem Mgmt).
  \item AI-assisted risk assessment for Changes.
  \item AIOps for monitoring and event correlation.
\end{itemize}
\item \lbl{Risks:} Data quality, bias, over-automation of human-critical decisions, loss of transparency; need clear governance and oversight.
\end{itemize}

\sectitle{Challenges in ITIL Adoption}
\begin{itemize}
\item Cultural and organisational resistance (shift from reactive to proactive).
\item Complexity and scope of ITIL 4 (34 practices – risk of “doing everything at once”).
\item Lack of executive sponsorship, funding or governance support.
\item Over-focus on process compliance instead of real value/experience.
\end{itemize}

%%%%%%%%%%%%%%%%%%%%%%%%%%%%%%%%
%%%% Practice Question Bank
%%%%%%%%%%%%%%%%%%%%%%%%%%%%%%%%
\hlhighlight{Practice Question Bank}
\begin{itemize}
  \item \lbl{1. Purpose of Integration Management:} Ensures all processes, deliverables, code modules, databases, and third-party tools align toward business goals.
  \item \lbl{2. Handling Scope Creep:} Apply change control, assess cost/time/resource impacts, communicate impacts, approve only if aligned with business value.
  \item \lbl{3. Why WBS is Used:} Breaks project into smaller tasks (frontend, backend, DB, testing) so responsibilities and progress tracking are clear.
  \item \lbl{4. IT Risk Example:} Vendor server downtime –> mitigate via redundancy, backup hosting, multi-zone cloud deployment.
  \item \lbl{5. QA Role:} Ensures deliverables meet requirements through testing, reviews, CI/CD pipelines.
  \item \lbl{6. Schedule Tools Benefit (CPM):} Identify tasks that cannot slip without delaying the project (critical path).
  \item \lbl{7. Stakeholder Mgmt Complexity:} Many groups with different technical literacy and goals –> alignment is harder.
  \item \lbl{8. Communication Importance:} Prevents misunderstanding, aligns technical/business teams, enables transparency and trust.
\end{itemize}

\hlhighlight{Practice Questions \& Answers}

\sectitle{IT Project Characteristics}
\begin{itemize}
\item \textbf{Define project; two distinguishing IT features}
\item A project is a temporary endeavour producing a unique product or service; IT projects are technology-intensive and experience higher uncertainty and rapid change.
\item \textbf{Why PM is essential for NovaSys}
\item PM ensures scope, schedule, and quality integration; provides structure for cross-border collaboration and stakeholder communication.
\item \textbf{Early action to align distributed teams}
\item Establish a shared collaboration plan – for example, an online RACI matrix or kickoff workshop clarifying roles, deliverables, and reporting cadence.
\end{itemize}

\sectitle{Project Lifecycle and Integration}
\begin{itemize}
\item \textbf{Five lifecycle phases}
\item Initiating –> Planning –> Executing –> Monitoring \& Controlling –> Closing, mapped to corresponding PMBOK process groups.
\item \textbf{Role of integration \& communication}
\item Integration aligns objectives, deliverables, and stakeholders; communication ensures visibility and coordination among teams.
\item \textbf{Governance to capture lessons learned}
\item Mandate a closure checklist requiring sign-off of a "lessons-learned register" reviewed by the PMO before project close.
\end{itemize}

\sectitle{Choosing the Right Methodology}
\begin{itemize}
\item \textbf{Waterfall vs Agile (key differences)}
\item Waterfall = linear, fixed scope, heavy documentation, limited stakeholder iteration; Agile = incremental, flexible scope, continuous collaboration.
\item \textbf{Why Hybrid suits BrightEdge}
\item Hybrid allows stable architecture under Waterfall while accommodating evolving reports via Agile sprints.
\item \textbf{Hybrid integration risk \& mitigation}
\item Risk = coordination failure between sequential and iterative teams; mitigate by defining clear sprint boundaries and shared configuration-management standards.
\end{itemize}

\sectitle{Scope Creep and Change Control}
\begin{itemize}
\item \textbf{Scope definition \& baseline purpose}
\item Scope defines approved deliverables; the baseline documents agreed scope, cost, and schedule to control change.
\item \textbf{Formal change-assessment mechanism}
\item Use a Change Control Board (CCB) and a structured change-request form assessing time–cost–quality impacts.
\item \textbf{Stakeholder communication strategy}
\item Hold a stakeholder workshop using SBAR (Situation-Background-Assessment-Recommendation) to present quantified trade-offs before approval.
\end{itemize}

\sectitle{Scheduling Delays and Recovery}
\begin{itemize}
\item \textbf{Schedule baseline purpose}
\item The baseline is the approved reference for performance tracking and variance analysis.
\item \textbf{Common dependency types with examples}
\item Finish-to-Start (coding –> testing) and Start-to-Start (training –> pilot setup).
\item \textbf{Corrective action to realign timeline}
\item Apply fast-tracking (overlapping configuration + training) or crashing after cost–benefit analysis to recover schedule.
\end{itemize}

\sectitle{Stakeholder Engagement and Communication}
\begin{itemize}
\item \textbf{Power–Interest classification}
\item Finance = high power / low interest –> keep satisfied; HR = low power / high interest –> keep informed.
\item \textbf{Engagement approaches}
\item Finance –> executive briefings focusing on ROI; HR –> interactive sessions and feedback loops.
\item \textbf{Tool/dashboard for transparency}
\item Use an integrated project dashboard (Microsoft Teams or Power BI) displaying sprint progress and issue logs.
\end{itemize}

\sectitle{Cost Estimation and Budget Control}
\begin{itemize}
\item \textbf{ROM vs definitive estimates}
\item ROM = ±50\% accuracy, used during initiation; definitive = ±10\% accuracy, used after scope baseline.
\item \textbf{Bottom-up estimating approach}
\item Break work packages into smaller activities, estimate individual costs, and aggregate for precision.
\item \textbf{Actions to maintain budget}
\item Value engineering and scope trade-offs; apply Earned Value Management to track CPI/SPI and trigger early corrections.
\end{itemize}

\sectitle{Earned Value Analysis}
\begin{itemize}
\item \textbf{Compute CPI and SPI}
\item CPI = EV/AC = 600/700 = 0.86; SPI = EV/PV = 600/900 = 0.67.
\item \textbf{Interpretation of indices}
\item With CPI < 1 and SPI < 1, the project is over budget and behind schedule.
\item \textbf{Recommended management action}
\item Review performance variances, identify causes, and prepare a corrective plan to improve cost and schedule efficiency before considering any re-baselining.
\end{itemize}

\sectitle{Quality Planning and Assurance}
\begin{itemize}
\item \textbf{Quality planning importance}
\item Quality planning sets standards and procedures to meet stakeholder requirements; vital for compliance and trust.
\item \textbf{Key QA activities}
\item Regular process audits and peer reviews of training datasets.
\item \textbf{Metric for data accuracy improvement}
\item Defect density or accuracy rate = (correct classifications / total cases) × 100\%.
\end{itemize}

\sectitle{Resource Allocation and Team Dynamics}
\begin{itemize}
\item \textbf{RACI for resource clarity}
\item RACI clarifies who is Responsible, Accountable, Consulted, and Informed for each task, avoiding duplication.
\item \textbf{Workload balancing strategies}
\item Use capacity-planning tools and rotate critical tasks via resource levelling.
\item \textbf{Motivational approach for cross-cultural teams}
\item Recognise team achievements publicly and offer shared virtual rewards to build trust and ownership.
\end{itemize}

\sectitle{Communication Breakdown and Stakeholder Expectations}
\begin{itemize}
\item \textbf{Causes of communication failure}
\item Unclear requirements and lack of feedback loops.
\item \textbf{Preventive role of communication plan}
\item A communication plan defines information needs, formats, frequency, and responsibilities for each stakeholder group.
\item \textbf{Tool and reporting technique}
\item Tool – Microsoft Teams or JIRA; reporting technique – weekly dashboards or RAG status reports.
\end{itemize}

\sectitle{Conflict and Resource Management}
\begin{itemize}
\item \textbf{Likely conflict type and cause}
\item Task-based conflict caused by role ambiguity and competing priorities.
\item \textbf{Conflict-resolution techniques}
\item Collaborating (Win–Win) and compromising (Find middle ground).
\item \textbf{Preventive measure}
\item Define clear role descriptions in a RACI matrix and hold regular alignment meetings.
\end{itemize}

\sectitle{Risk Identification and Assessment}
\begin{itemize}
\item \textbf{Purpose and content of risk register}
\item A risk register lists identified risks, causes, probability, impact, owners, and planned responses.
\item \textbf{Qualitative vs quantitative analysis}
\item Qualitative = subjective ranking using probability/impact matrix; quantitative = numerical estimation of exposure (e.g., Monte Carlo).
\item \textbf{High-priority risks and responses}
\item (1) Compliance change –> mitigate by early gap assessment and controls. (2) API instability –> transfer via redundant vendor or service agreement with penalties.
\end{itemize}

\sectitle{Risk Response Planning and Monitoring}
\begin{itemize}
\item \textbf{Objectives of risk response plan}
\item To specify actions, owners, and timing for each major risk, reducing threats and exploiting opportunities.
\item \textbf{Role of contingency and fallback plans}
\item Contingency plan = actions if risk occurs; fallback = backup if contingency fails.
\item \textbf{Metric for risk effectiveness}
\item Track "open high-severity risks" per month or \% of risks with active response plans.
\end{itemize}

\sectitle{Procurement Strategy and Contract Selection}
\begin{itemize}
\item \textbf{Fixed-price vs T\&M (risk \& cost)}
\item Fixed-price = vendor bears cost risk; T\&M = buyer bears cost risk but flexible for change.
\item \textbf{Factors for contract choice}
\item Requirement clarity and vendor maturity.
\item \textbf{Recommended model for CyberLink}
\item T\&M is better for evolving requirements; include rate caps and clear change-control clauses.
\end{itemize}

\sectitle{Procurement Planning and Bid Evaluation}
\begin{itemize}
\item \textbf{Procurement planning steps}
\item Define needs –> prepare procurement documents –> conduct bidding –> evaluate –> award –> manage contract.
\item \textbf{Weighted scoring model benefit}
\item Weighted model assigns scores to criteria (SLA, cost, security) for objective comparison.
\item \textbf{Top-weight criteria}
\item Data security and SLA uptime guarantees.
\end{itemize}

\sectitle{Contract Administration and Vendor Performance}
\begin{itemize}
\item \textbf{Contract administration activities}
\item Monitor deliverables against SLA and manage contract changes.
\item \textbf{Change control under fixed-price}
\item All scope changes require formal change request and cost/schedule review before approval.
\item \textbf{Negotiation approach for disputes}
\item Use principled negotiation (focus on interests not positions); agree on shared facts before commercial adjustment.
\end{itemize}

\sectitle{Risk and Procurement Integration}
\begin{itemize}
\item \textbf{Why procurement risks belong in risk register}
\item Supplier delays and contract breaches directly impact schedule and budget; they need visibility in risk tracking.
\item \textbf{Risk transfer techniques}
\item Insurance and performance bonds / penalty clauses.
\item \textbf{Vendor-risk monitoring method}
\item Conduct monthly vendor scorecards or joint risk reviews with procurement and PMO.
\end{itemize}

\sectitle{Leadership Styles and Team Motivation}
\begin{itemize}
\item \textbf{Leadership style comparison}
\item Directive = close control / clarity; Participative = shared decisions; Transformational = inspire vision \& creativity.
\item \textbf{Best style for TechFusion}
\item Transformational-participative blend fits knowledge projects—empowers experts yet keeps focus.
\item \textbf{Immediate motivational technique}
\item Implement recognition boards or innovation awards; celebrate sprint achievements to rebuild engagement.
\end{itemize}

\sectitle{Governance Frameworks and Decision Escalation}
\begin{itemize}
\item \textbf{IT governance purpose}
\item Governance = structures \& processes for decision-making, ensuring projects align with strategic goals and compliance.
\item \textbf{Key governance components}
\item Steering committee with defined authority; clear reporting hierarchy with decision thresholds.
\item \textbf{Escalation mechanism}
\item Use tiered escalation matrix—PM –> Program Manager –> CIO—triggered by cost/scope variance limits.
\end{itemize}

\sectitle{Ethical Leadership and Stakeholder Trust}
\begin{itemize}
\item \textbf{Ethical dilemma}
\item Pressure to misrepresent data integrity for schedule compliance.
\item \textbf{Ethical leadership guidance}
\item Ethical leaders prioritise honesty, transparency, and stakeholder trust; follow organisational code of conduct.
\item \textbf{Immediate steps for resolution}
\item Escalate to governance board, document issue, halt release until compliance achieved; protect whistle-blower team members.
\end{itemize}

\sectitle{Integration Across Knowledge Areas}
\begin{itemize}
\item \textbf{Purpose of Integration Management}
\item Integration Management ensures all knowledge areas (scope, cost, risk, quality) work coherently toward project objectives.
\item \textbf{Tools for cross-project coordination}
\item Use integrated master schedule (IMS) and dependency matrix.
\item \textbf{Reporting artefact for integrated performance}
\item Develop a consolidated project dashboard combining CPI/SPI from all sub-projects.
\end{itemize}

\sectitle{Change Control in Integration Management}
\begin{itemize}
\item \textbf{Change-control process}
\item Receive –> Evaluate –> Approve / Reject –> Update plans –> Communicate –> Implement.
\item \textbf{Importance of impact analysis}
\item Prevents cascading cost/schedule issues; maintains baseline integrity.
\item \textbf{Tool to track approved changes}
\item Use integrated change-log system (e.g., JIRA workflow or PMIS dashboard).
\end{itemize}

\sectitle{Transitioning from Projects to Services}
\begin{itemize}
\item \textbf{Project vs service management}
\item Project management = temporary delivery of change; Service management = continuous value delivery post-implementation.
\item \textbf{Actions for smooth transition}
\item Conduct formal service-transition handover workshop and update service catalogue with support contacts/SLA ownership.
\item \textbf{KPI for successful transition}
\item "First-Contact Resolution Rate" or "\% of tickets handled by service desk without escalation."
\end{itemize}

\sectitle{ITIL v4 Guiding Principles and Service Value System}
\begin{itemize}
\item \textbf{Purpose of the Service Value System}
\item SVS = framework ensuring organisation components (workflows, governance, practices) co-create value through services.
\item \textbf{Guiding principles impact}
\item "Focus on value" aligns decisions to customer outcomes; "Collaborate and promote visibility" breaks silos via shared dashboards.
\item \textbf{Measure improvement-culture maturity}
\item Run quarterly CSAT surveys or track \# of improvement initiatives implemented per quarter.
\end{itemize}

\sectitle{Incident and Problem Management in Practice}
\begin{itemize}
\item \textbf{Incident vs problem management}
\item Incident = restore service ASAP; Problem = identify and eliminate root cause.
\item \textbf{Value of linking practices}
\item Linking ensures temporary fixes trigger permanent solutions and knowledge-base updates.
\item \textbf{Metric for long-term stability}
\item Mean Time Between Incidents (MTBI) or repeated-incident rate.
\end{itemize}

\sectitle{Change Enablement and Risk Control}
\begin{itemize}
\item \textbf{Purpose of change enablement}
\item To balance need for change with risk control by evaluating, authorising, and documenting changes.
\item \textbf{Risk-reduction techniques in change enablement}
\item CAB (Change Advisory Board) review and post-implementation evaluation.
\item \textbf{Reporting tool for change transparency}
\item Implement ServiceNow or JIRA Change Calendar for live visibility of change status.
\end{itemize}

\sectitle{Continual Improvement and Automation}
\begin{itemize}
\item \textbf{How automation supports continual improvement}
\item Automation reduces manual repetition, enabling staff to focus on analysis and service enhancement.
\item \textbf{Automation risks and mitigations}
\item Risks = resistance to change \& AI bias; mitigate via early engagement and algorithm testing.
\item \textbf{Training initiative for transition}
\item Launch an "AI for Service Professionals" training program linking new skills to career progression.
\end{itemize}

\sectitle{Linking Project Management and ITSM}
\begin{itemize}
\item \textbf{PM vs ITSM objectives}
\item PM = temporary change delivery; ITSM = ongoing service stability; together they complete the value chain.
\item \textbf{Integration points between PM and ITIL}
\item Integration via release \& deployment planning and shared change-management boards.
\item \textbf{Governance mechanism for alignment}
\item Establish joint PMO-Service Governance Forum with shared KPIs (availability vs project benefits).
\end{itemize}

%%%%%%%%%%%%%%%%%%%%%%%%%%%%%%%%
%%%% WEEK 13 – EXAM TIPS
%%%%%%%%%%%%%%%%%%%%%%%%%%%%%%%%

\hlhighlight{Exam Review Notes}

\sectitle{Answer Structure}
\begin{itemize}
\item Define key concept(s) clearly.
\item Explain mechanism/why it matters.
\item Apply to scenario (case details, trade-offs).
\item Conclude with recommendation/summary.
\end{itemize}

\sectitle{Good Exam Practice}
\begin{itemize}
\item \concept{Read carefully}
\item \concept{State assumptions}
\item \concept{Plan your answer}
\item \concept{Apply theory to the scenario}
\item \concept{Discuss trade-offs}
\item \concept{Show evidence}
\item \concept{Be structured}
\item \concept{Timebox and review}
\item \concept{Conclude with recommendation}
\end{itemize}

\end{multicols}
\end{document}